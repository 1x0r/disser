\chapter*{Введение}
\addcontentsline{toc}{chapter}{Введение}	% Добавляем заголовок в оглавление

Данная работа посвящена исследованию метода грамматической эволюции применительно к задаче синтеза системы управления.

Поиск решений многих инженерных задач требует, с одной стороны, творческого подхода, с другой -- поиска оптимального или близкого к оптимальному конструкторского решения среди множества схожих структур.
И если талант инженера-человека пока невозможно алгоритмизировать (и, соответственно, написать имитирующую его программу), то вторая, поисковая, сторона инженерного дела стала доступна для моделирования прежде всего различными интеллектуальными численными методами.
В последнее время, благодаря взрывному росту и повсеместной доступности вычислительных мощностей, эвристические и метаэвристические переборные алгоритмы из нишевой области искусственного интеллекта для решения ограниченного круга задач стали поистине универсальным инструментом, достойным находиться в арсенале каждого инженера.

Задача синтеза управления -- одна из основных задач теории автоматического управления~\cite{Дивеев2011}.
Технически синтезом управления можно считать введение в уравнения, описывающие движение объекта, в общем случае нелинейной управляющей функции, зависящей от состояния объекта.
Эта функция должна как удовлетворять определенным техническим ограничениям, так и доставлять минимум каждому из требуемых критериев качества.
Если функция управления введена аддитивно, то можно считать, что система охвачена обратной связью по состоянию.

Синтез управления может быть:
\begin{itemize}
    \item структурным, при котором ищется структура управляющей функции (обычно в виде математического выражения или структурной схемы);
    \item параметрическим, при котором в заданной структуре системы управления подбираются значения свободных параметров;
    \item структурно-параметрическим, являющимся комбинацией структурного и параметрического синтеза.
\end{itemize}
Исключительно структурный синтез практически не встречается.
В данной работе рассматривается структурно-параметрический синтез, как наиболее актуальный.

Существует три основных подхода к синтезу систем управления:
\begin{itemize}
    \item графо-аналитический -- большая часть методов синтеза, основанных на формировании желаемых ЛАЧХ системы, относится к графическим или графически-аналитическим; к графическим методам также относится метод синтеза, основанный на использовании корней характеристического уравнения;
    \item аналитический -- наиболее разработанный подход, <<столпами>> которого являются синтез на основе функции Ляпунова, динамическое программирование Беллмана, аналитическое конструирование оптимальных регуляторов (АКОР), аналитическое конструирование агрегированных регуляторов и бэкстеппинг (от англ. \textit{backstepping} -- пошаговый отход назад);
    \item численный -- наиболее молодой подход; параметрический синтез численными методами получил в в последние десятилетия широкое распространение, структурно-параметрический же не так давно начал набирать популярность благодаря появившимся методам символьной регрессии.
\end{itemize}

Графо-аналитические методы, такие как корневой метод~\cite{Соколов1952}, метод корневого годографа~\cite{Kuo1967}, выбор структуры САУ на основе желаемых логарифмических амплитудно-частотных характеристик~\cite{Солодовников1953} и другие, хотя и наглядны, и достаточно просты, не являются перспективным направлением. 
Несмотря на то, что внедрение вычислительных технологий дало толчок в развитии этих методов~\cite{Барковский1981, Имаев1998}, они чрезвычайно трудоемки и, в основном, работают только с линейными системами. Еще одной проблемой является их плохая масштабируемость: с увеличением сложности объекта управления заметно растет и объем необходимых вычислений.

Аналитические методы имеют основательный теоретический базис, используются в науке и производстве и хорошо себя зарекомендовали при решении задач синтеза. Основными аналитическими методами синтеза являются линейно-квадратичная оптимизация, модальное управление, АКОР, АКАР, бэкстеппинг.

% синтез по Ляпунову
А.М. Ляпунов в работе~\cite{Ляпунов1935} положил начало развитию аналитических методов синтеза, подробно исследовав понятие устойчивости и введя концепцию функций Ляпунова.
Идея синтеза на основе функций Ляпунова сводится к стабилизации системы в окрестности некоторого целевого состояния.
Один из первых подходов к синтезу управления, получивший название линейно-квадратичной оптимизации, рассмотрен в 1960 году Р.\,Э. Калманом в работах~\cite{Kalman1960, Kalman1960a} для систем с непрерывным и дискретным временем.
В том же году независимо от Калмана отечественный ученый А.\,М. Летов разработал~\cite{Летов1960,Летов1960a,Летов1960b,Летов1961,Летов1962} аналогичный метод стабилизации линейных стационарных систем при условии выполнения требований квадратичности функционала качества -- аналитическое конструирование (оптимальных) регуляторов (АКОР).
Ограничения метода включают требования квадратичности функционала и линейности объекта.

% уравнение Беллмана
Одним из самых значимых результатов в области синтеза систем управления является уравнение Беллмана~\cite{Bellman1957, Болтянский1968}, являющееся основой оптимизационного метода динамического программирования. 
Развитием метода является уравнение Гамильтона -- Якоби -- Беллмана в частных производных~\cite{Кружков1975}.

% прочий аналитический синтез
А.А. Фельдбаум разработал метод фазового пространства~\cite{Фельдбаум1953, Фельдбаум1955} для решения задачи синтеза для линейных систем второго порядка.

% Заключение про линейный аналитический синтез + вступление к части про нелинейный
Рассмотренные выше методы синтеза обладают одним общим ограничением -- модель объекта управления должна быть линейной. 
В конце XX -- начале XXI века было предложено несколько новых принципов синтеза систем управления по состоянию.
Среди них стоит выделить аналитическое конструирование агрегированных регуляторов (АКАР), предложенное А.\,А. Колесниковым~\cite{Колесников2006, Колесников1994}, и бэкстеппинг, разработанный в 1980-1990-х годах И. Канеллакопулосом, П. Кокотовичем и др.~\cite{Kanellakopoulos1989, Kanellakopoulos1991, Krstic1995}.

% АКАР
Метод АКАР основан на принципах синергетической теории управления: формируется расширенная система, включающая в себя не только описание структуры объекта и управляющих воздействий, но и задающие и возмущающие воздействия внешней среды.
Синтезируемое управление вначале приводит изображающую точку системы на целевое притягивающее многообразие (пересечение притягивающих многообразий в векторном случае), а затем обеспечивает движение вдоль этого многообразия (пересечения многообразий). 
Метод опирается на выбор инженером, проектирующим систему управления, формы многообразия (многообразий), опираясь на собственный опыт и внутренние свойства системы. Метод становится достаточно трудоемким при повышении степени системы.

% бэкстеппинг
Бэкстеппинг 

% эволюционные методы синтеза
В то же время с развитием вычислительной техники популярность приобрели численные методы.
Прежде всего, их польза заключается в отсутствии необходимости проведения натурных экспериментов, зачастую дорогостоящих, так как все необходимые исследования инженер может проводить на компьютере.
До недавнего времени область применения численных методов для задач синтеза была ограничена настройкой параметров управляющих систем известной структуры, например, для ПИД-регулятора.
Но в последние годы распространение получили численные методы поиска символьных выражений, основанные на эволюционных алгоритмах.

% генетическое программирование
В начале 1990-х годов Дж.~Коза предложил использовать генетические алгоритмы, разработанные в 1970-х Дж.\,Х.~Холландом~\cite{Holland1973, Holland1975}, для построения компьютерных программ~\cite{Koza1990, Koza1990a, Koza1992a}.
Метод получил название <<генетическое программирование>> (\textit{genetic programming}, GP, ГП) и стал основой для нескольких других методов поиска.
Впоследствии он был адаптирован для поиска и оптимизации произвольных структур: электрических схем~\cite{Wang2007}, свойств строительных материалов~\cite{Gonzalez-Taboada2016} и многих других задач.

Генетическое программирование осуществляет поиск в пространстве строковых выражений возможных решений поставленной задачи.
Строковое выражение формируется в виде S-выражения~\cite{McCarthy1960}.
S-выражения -- элементарные строковые вычислительные конструкции, используемые, к примеру в языке программирования LISP. 
Они имеют вид 
\begin{equation}
    \left(operator\ operand\ operand\ operand\ ...\right),
    \label{intro:s_expression}
\end{equation}
где операндами могут выступать, в свою очередь, S-выражения.
Такая инфиксная конструкция выражений упрощает их синтаксический разбор.
S-выражение можно представить в виде дерева, в котором корневым узлом будет оператор, а поддеревьями -- операнды.
Над S-выражениями в ГП вводится оператор обмена поддеревьями.

В первые же годы появилась идея использования ГП для задач управления.
Дж.~Коза в~\cite{Koza1992} стабилизировал обратный маятник на тележке.
Льюис и Фэгг~\cite{Lewis1992} описали процесс конструирования искусственной нейронной сети для управления шагающим роботом.
В работах Нордина и Банжафа~\cite{Nordin1996, Nordin1997} предложены методы онлайн-управления мобильным роботом на основе генетического программирования.

Однако применение ГП для решения задачи для синтеза осложняется наличием оператора обмена поддеревьями между возможными решениями. 
К проблеме неограниченного роста объема решений (\textit{bloating}) существуют попытки подступиться~\cite{Trujillo2016}, однако решить её полностью не удалось.
Кроме того, в ГП невозможно структурировать и, тем самым, ограничить пространство поиска решений.

% аналитическое программирование
\textbf{Аналитическое программирование}
Автор - Зелинка


% 
\textbf{Грамматическая эволюция}
Предллжена Онилом и Райаном
плюсы
в гибкости задания структур
в возможности введения сложных управляющих конструкций в состав программ
в возможности написать DSL под задачу

минусы
в необходимости трансляции/инерпретации
в позиционности кода решения
в вероятной невалидности выражений

% сетевой оператор
\textbf{Метод сетевого оператора} (СО) разработал А.\,И.~Дивеев в середине первого десятилетия XXI века~\cite{Дивеев2006}.
Код символьного выражения в методе СО представлен целочисленной верхнетреугольной матрицей специального вида, матрицей сетевого оператора, построенной на основе матрицы смежности направленного ациклического графа.
В графе символьного выражения унарные функции сопоставлены дугам, а бинарные операторы -- узлам.
Бинарные операторы должны отвечать свойствам коммутативности и ассоциативности, а также иметь единичный элемент.
Метод СО позволяет находить только такие выражения, в состав которых входят исключительно функции одного аргумента и бинарные коммутативные операции.
Матрица сетевого оператора, упорядоченные множества унарных функций и бинарных операторов, множество входных переменных и констант и номер строки матрицы, соответствующей узлу-стоку графа символьного выражения задают полное описание символьного выражения.

Метод сетевого оператора построен на принципе базисного решения: выбирается начальное символьное приближение к решению и кодируется матрицей сетевого оператора.
К матрице последовательно применяются малые вариации из упорядоченного множества малых вариаций, соответствующих малым изменениям графа символьного выражения.
Поисковым алгоритмом служит модифицированный стационарный генетический алгоритм.
Поиск проводится на множестве упорядоченных множеств малых вариаций.

Метод сетевого оператора успешно применялся для задач синтеза управления \todo{вставить ссылки здесь и далее}, в том числе и логико-функционального~\cite{ДивСофр2012, АтиенДивеев2012}, идентификации систем~\cite{Дивеев2008a}, аппроксимации функций.

\todo{плюсы сделать связными}
Базисное решение, заданное в виде матрицы, не требует введения сложных структур данных для хранения решения: достаточно использования любого современного языка, поддерживающего двумерные массивы.
В отличие от метода аналитического программирования и некоторых других вычислительных методов поиска символьных выражений, матрица сетевого оператора взаимно однозначно отображается в символьное выражение.
Принцип базисного решения дает инженеру по проектированию систем управления возможность использовать опыт эксперта при построении решения.
Сравнительно недавно появились работы по исследованию каскадов матриц сетевого оператора.
Такой каскад называется многослойным сетевым оператором.
Многослойный СО структурно схож с искусственными нейронными сетями.
Сетевой оператор существенно отличается от других методов потенциально многократным использованием промежуточных выражений.
Каждый узел графа сетевого оператора хранит некоторое символьное выражение.
Так как из одного узла может выходить несколько дуг, то это выражение может входить в итоговое выражение неоднократно.

Однако метод сетевого оператора не лишен и недостатков.
Семейство используемых функций ограничено унарными, а на бинарные операторы наложены существенные ограничения.
Как следствие, логические операторы и функции с количеством аргументов большим 2 не используются в поиске выражений.
Были успешные попытки введения логических функций за счет внешней структуры -- логического сетевого оператора~\cite{ДивСофр2012, АтиенДивеев2012}, однако такой подход существенно удорожает поиск решения.

% декартово программирование
\textbf{Декартово программирование} -- ещё один 

% эволюция матрицы парсинга (?)
Хронологически последним из существующих методов символьной регрессии был разработан метод эволюции матрицы разбора выражений~\cite{Luo2012} (\textit{parse-matrix evolution, PME}).
Как и в методе сетевого оператора для представления графа, описывающего символьное выражение, используется целочисленная матрица.
Отличие между методами прежде всего в том, что в PME в эволюционном процессе используется сама целочисленная матрица, а в МСО -- вариации некоторой базисной матрицы.
Для преобразования матрицы в символьное выражение в PME используется таблица правил преобразования численных значений в символьные выражения.
Авторы PME указывают на сравнительную простоту декодирования выражений по сравнению с ГЭ.
Отсутствие замен нетерминальных символов на нетерминальные в таблице правил сокращает количество обращений к ней и уменьшает длину кода решений.
Из существенных недостатков метода можно отметить, во-первых, сложность или невозможность введения программных управляющих структур, таких как циклы, условные конструкции, рекурсивные функции, и, во-вторых, увеличение матрицы кода решения на $p+2$ столбца при добавлении функций с $p$ аргументами в таблицу правил преобразований.

Решение задач в области управления методами символьной регрессии требует существенных вычислительных ресурсов.
Интерес к использованию методов символьной регрессии возродился во втором десятилетии XXI века: вычислительные возможности персональных компьютеров достигли тех значений, при которых возможно проводить полноценное численное моделирование сложных систем за разумное время.
Применение символьной регрессии к задачам управления подразумевает многократный запуск процесса моделирования требовательных к скорости вычислений программ (к примеру, решения системы ОДУ высокого порядка).

Общими проблемами методов символьной регрессии в инженерных задачах являются переобучение, преждевременная сходимость множества решений к локальному оптимуму, высокие вычислительные затраты.

\textbf{Актуальность исследования}

В настоящее время все аналитические методы, разработанные для решения задачи синтеза, не являются универсальными. Во всех методах накладываются ограничения либо на вид системы, либо на функционал. Кроме того, основные подходы -- бэкстеппинг, АКОР, АКАР -- опираются на выбор экспертом некоторой функции, на базе которой строится решение задачи синтеза.

\textbf{Объект исследования}

\todo {Объектом исследования может быть либо объект управления, либо сам алгоритм. Что?}

\textbf{Основные положения, выносимые на защиту}

    \begin{itemize}
        \item Разработка языка построения систем управления в нормальной форме Бэкуса---Наура.
        \item Применение метода грамматической эволюции к задаче синтеза управления для динамической системы.
        \item Решение задачи структурно-параметрического синтеза системы управления \todo{динамическим объектом}.
        \item Программный комплекс, реализующий метод грамматической эволюции для задачи синтеза системы управления \todo{динамическим объектом}.
    \end{itemize}

\textbf{Цель и задачи диссертационной работы}

В данной диссертационной работе ставились следующие цели и задачи.
    \begin{itemize}
        \item Рассмотреть существующие аналитические и численные методы синтеза систем управления.
        \item Разработать численные методы и алгоритмы решения задачи струкутрно-параметрического синтеза системы управления динамическим объектом.
        \item Разработать и реализовать эффективные программные средства для численного структурно-параметрического синтеза.
        \item Провести вычислительный эксперимент, 
    \end{itemize}

\textbf{Научная новизна}

В диссертационной работе получены следующие результаты, обладающие научной новизной.
    \begin{itemize}
        \item Сформулирована постановка задачи синтеза управления 
        \item Разработан основанный на методе грамматической эволюции метод структурно-параметрического синтеза системы управления динамическим объектом, заданным в виде системы дифференциальных уравнений. 
        \item 
    \end{itemize}

\textbf{Теоретическая значимость работы}

Теоретическая значимость работы заключается в аналитическом исследовании применимости метода грамматической эволюции для решения задачи синтеза системы управления.

\textbf{Практическая значимость работы}

\todo{В чем практическая значимость? В высвобождении человекочасов инженернов-проектировщиков систем управления}

\textbf{Методология и методы исследования}

Методологическую основу исследования составлял вычислительный эксперимент.
При выполнении диссертационной работы применялись понятия, концепции и методы дискретной математики (теории множеств, теории графов), информатики, теории автоматического управления, теории оптимального управления, теория формальных языков. При разработке и реализации программных средств проведения вычислительных экспериментов использовался свободно распространяемый язык Python 3.5 с библиотеками научных вычислений NumPy 1.11.0, SciPy 0.17.0, sklearn 0.17.1, matplotlib 1.4.3, а также с библиотекой параллельных вычислений joblib 0.9.4. 

\textbf{Предмет исследования}

Предметом исследования работы является применение вычислительного метода грамматической эволюции для решения задачи структурно-параметрического синтеза системы управления.

\textbf{Публикация результатов исследования}

Основные положения диссертационной работы доступны в 14 публикациях~\cite{DanDiKaSo2015,DivKazSof2014,DivKazSof2013,DivKazSof2013a,ДанДиКаСо2014,ДивееКаза2013,ДивееКаза2012,Казарян2013,Казарян2014,Казарян2014a,КазКулЖар2010,КазХамКоч2014,КулЖарКаз2010}, из них:
\begin{itemize}
    \item 1 входит в наукометрическую базу данных ISI Web of Knowledge;
    \item 5 входят в наукометрическую базу данных Scopus;
    \item 3 опубликованы в изданиях, входящих в перечень ВАК на момент публикации;
    \item 8 входят в РИНЦ.
\end{itemize}

В совместных публикациях доли участия соавторов в исследованиях равны.

\textbf{Апробация результатов исследования}

Основные положения диссертации были представлены и обсуждены на:
\begin{itemize}
    \item X международном симпозиуме «Интеллектуальные системы» (INTELS'2012), Вологда, 25--29 июня, 2012 г.;
    \item VI международной научно-практической конференции «Инженерные системы — 2013», Москва, 24--26 апреля 2013 г.;
    \item $8^{\text{th}}$ IEEE Conference on Industrial Electronics and Applications (ICIEA), Melbourne, VIC, Australia, 19--21 June, 2013;
    \item $21^{\text{st}}$ Mediterranean Conference on Control \& Automation (MED'13), Platanias-Chania, Crete, Greece, June 25--28, 2013;
    \item VII Всероссийской научно-практической конференции «Инженерные системы — 2014», Москва, 16--18 апреля 2014 г.;
    \item $22^{\text{nd}}$ Mediterranean Conference on Control \& Automation (MED'14), Palermo, Sicilia, Italy, June 16--19, 2014;
    \item XI международном симпозиуме «Интеллектуальные системы» (INTELS'2014), Москва, 30 июня--4 июля 2014 г.;
    \item VIII Международной научно-практической конференции «Инженерные системы — 2015», Москва, 16--18 апреля 2014 г.;
    \item ежеквартальных семинарах кафедры кибернетики и мехатроники инженерного факультета ФГАОУ ВО <<Российский университет дружбы народов>> (РУДН).
\end{itemize}

В ФГБУ <<Федеральный институт промышленной собственности>> получено свидетельство о государственной регистрации программы для ЭВМ №2014614355 «Идентификация математической модели динамической системы методом грамматической эволюции» от 22 апреля 2014 г.

\textbf{Структура и объем работы}

Диссертация состоит из введения, \todo{???? глав}, приложения и списка литературы. Диссертация содержит \todo{87 страниц текста из 126, включает 30 рисунков, 2 таблицы}. Список литературы содержит \todo{51 наименование}.

\clearpage