\chapter{Задача синтеза системы управления динамическим объектом} \label{chapter:synthesis_problem}

В работе рассматривается задача синтеза управления некоторым объектом.
Модель объекта задаётся в форме системы в общем случае нелинейных дифференциальных уравнений, в которые входит функция управления, зависящая от переменных состояния.
Синтезированная функция управления должна, во-первых, удовлетворять ограничениям, заданных обычно как для значения управления, так и для переменных состояния, во-вторых, минимизировать некоторый критерий (или множество критериев) качества, определённых в рамках задачи.

\section{Методы синтеза систем управления} \label{sect:synthesis_methods}

\subsection{Аналитические методы синтеза управления}

\subsection{Численные методы синтеза управления}

\section{Формальная постановка задачи синтеза системы управления} \label{sect:formal_synthesis}

\clearpage
